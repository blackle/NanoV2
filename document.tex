\documentclass[a5paper,10pt]{book}
\usepackage[T1]{fontenc}
\usepackage[utf8]{inputenc}
\usepackage{lmodern}
\usepackage{enumerate}
\usepackage{etaremune}
\usepackage[english]{babel}
\usepackage{amsmath}
\usepackage{amsfonts}
\usepackage{graphicx}
\usepackage{setspace}
\usepackage{wrapfig}
\usepackage[bitstream-charter]{mathdesign}
%\usepackage{baskervald}
\usepackage[T1]{fontenc}
\usepackage{fullpage}
\usepackage{everyhook}

\newcommand{\interlude}{\begin{center}
\line(1,0){250}
\end{center}}

\def\changemargin#1#2{\list{}{\rightmargin#2\leftmargin#1}\item[]}
\let\endchangemargin=\endlist
\newenvironment{dialogue}
{
\it
\renewcommand{\:}{\rm:}
\PushPreHook{par}{\it}

\begin{changemargin}{ 0.4 in}{ 0.4in}
\addtolength{\parskip}{0.1em}
}
{
\addtolength{\parskip}{-0.1em}
\end{changemargin}

\PopPreHook{par}
}

%fix typewriter no break
\newcommand*\justify{%
  \fontdimen2\font=0.4em% interword space
  \fontdimen3\font=0.2em% interword stretch
  \fontdimen4\font=0.1em% interword shrink
  \fontdimen7\font=0.1em% extra space
  \hyphenchar\font=`\-% allowing hyphenation
}

%\includeonly{ch3/document}

\begin{document}
\title{Waystation} 
\author{Isabelle Knott}
\date{November, 2015}
\maketitle{}

\tableofcontents 

\chapter{}

Terry and I were sitting together on the bench in the control room, and by ``control room'' I mean a utility shed with a bunch of high-tech equipment stuffed inside. On the desk in front of us were our thinkpads, Linux on mine and Windows on Terry's. The thinkpads were connected by serial cable--actually, a serial to usb cable--to an absurdly large stack of data multiplexers, maybe 8 or 9 of them, starting on a stool near our feet and towering up to about human height. In all they permitted access to 64 lensless DSLR cameras facing straight up. Actually, that's a lie, they do have a ``lens,'' but I'll get to that in a moment.

Other than the tech the shed was pretty bare. There were a few toolkits on the floor filled with all sorts of screwdrivers and pliers. To the right of us on the wall was a big analogue clock with missing hands and a tiny watch-sized digital version glued to the face. Behind us was a big map of the surrounding area--just wide enough to include Urumqi--with a red pushpin indicating the location of the observatory.

Also on the floor was a mess of power and data cables for both the multiplexers and the laptops, which Anne almost tripped over coming in.

\begin{dialogue}
  Anne\: Has 16 started working again? We jimmied the cables around.

  Terry\: Uh, give me a minute to check.. Maggie?
\end{dialogue}

That was my cue to manually configure the multiplexer stack to route the data from camera \#16 to my computer, where I could run diagnostics on it. Since the DSLR's failure mode is ``don't record anything if something is wrong'' here ``diagnostics'' mean ``take a photo.'' The ``something wrong'' could range from ``corrupted disk'' to ``scratch on CCD.'' We were /hoping/ the problem would be ``bad connection,'' since this was the problem for cameras 18, 24, and 33.

I sent the photo request to 16 and waited for the photo to come back. The exposure of the camera was configured so a signal of ``the light + light pollution'' would map to full intensity on the imaging sensor, with a low ISO for an adequate signal to noise ratio. Since the light is dim this implied an exposure time of 2 seconds.

At our latitude the speed of the earth at the surface is about 800 kilometers an hour, which is the same speed that a star shadow travels westward. In 2 seconds a shadow would have traveled a little under half a kilometer. If we tried to image the anomaly with these exposure settings (the only feasible exposure settings, as it happens) all we'd have is an extremely horizontally blurred image.

In fact, this is what we've had for a very long time until the Goldin Orbital Shadow Imaging Platform was put into rendezvous with it. The images it captured, only about a meter per pixel, gave us the first look of the anomaly's dented teardrop shape. No other imaging experiment has gotten better resolution than Goldin. Hopefully this streak will end with us.

The solution to the blurring problem is fairly simple. When I said I was lying about the cameras having lenses, this was because they /do/ have lenses, but they're not anything you'd picture in your head when I say ``DSLR Lenses.'' In fact, they look more like something we bought from a lighthouse supply store.

This is our setup: To start, we have an eight by eight matrix of lensless DSLR cameras pointing straight up. Above each camera is a $1\mathrm{m}^2$ Fresnel lens that focuses collimated light (/the/ light, in our case) into the DSLR's $7\mathrm{mm}^2$ CCD sensor. This multiplies the total light received by each camera 22-thousand-fold, allowing us to reduce our exposure time to only a tenth of a millisecond. In that time a shadow can only move 2 centimeters. Although this movement is above the per-pixel resolution of our cameras, it is /also/ below noise level. Even if it were above noise level, we could deconvolve it fairly easily--small motion blur behaves very nicely in the frequency domain.

My request to camera 16 failed to bounce back. I turned to Terry and shook my head.

\begin{dialogue}
  Terry\: Well fuck.

  Anne\: Hmmmm, suppose we need to replace it with a backup?

  Maggie\: Didn't we already use both of them for 2 and 44?

  Terry\: We did...

  Anne\: Oh no. Should we try fixing the three and use whichever works?

  Maggie\: The three?

  Anne\: The broken 2, 44, and 16s.

  Maggie\: Oh, yeah. Ugh, we can make do for now with 63 since we're just snapping one of the orbcomms tonight. When is that happening again?

  Terry\: Like, forty minutes.

  Maggie\: Oh shit. Ok, let's just let it be for now. We can't risk fucking something up and missing our dry runs. Speaking of dry runs, we should really be get on that, Terry. 
\end{dialogue}

Terry nodded and alt+tabbed to a .NET program he'd put together that automatically requests, downloads, and splices the 64 images together, then does some fancy post-processing on it to generate a ``shadow image'' and a ``confidence image.'' It's actually really spiffy, each image appears in it's own little box as it arrives and goes through post-processing separately, appearing in two different little boxes on the other side of the window. Once all images have been collected the borders between them disappear and you can export them as pngs or jpegs or csv files.

it crashed the first time we tried to run it.

The problem was the most obvious one: the program was expecting camera 16 to respond to the request with an image, but instead it returned with an error code (always 1, which is extremely unhelpful.) It wouldn't see the error but instead see an empty output, try to parse it as a RAW stream, then crash because the resolution of the image is 0x0 when all the functions assume a resolution of 4000x3000.

\begin{dialogue}
  Terry\: Well, I wasn't planning on this happening.

  Maggie\: We can always use my script that just does the data dumps, then put those through your app later to get the pub images.

  Terry\: Uh, I'd rather we fix it now so we have something to show Leland by dinner, haha. It doesn't seem to hard to fix...
\end{dialogue}

Which was the opposite of what happened. We first tried to evaluate what it would take to add the correct error handling. It wasn't pretty, most of the code assumed all 64 image slots would be filled, from the confidence calculator to the image splicer. We then opted to add a simple error handler that loads dummy data of all zeros into the missing frame. This crashed too, division by zero error.

The final solution was not an elegant one. If there is an error, clone the output from the first camera. Terry and I cringed when we came up with the workaround, then dramatically sobbed when it worked flawlessly.

We did about eight dry runs, picking up one or two pixel sized shadows below noise floor (the confidence maps said ``< 0.01" all around.) We called Anne and Frieda inside to show them our results.

\begin{dialogue}
  Maggie\: Ta-da!

  Anne\: Looks good guys! I see we picked up some strays.
\end{dialogue}

Annie pointed to a lone white pixel on Terry's screen.

\begin{dialogue}
  Maggie\: I can look up what that was.

  Terry\: If we're gonna assume it /is/ something doesn't that defeat the purpose of the confidence maps?

  Frieda\: Oh hush, I specifically made the error estimator extremely conservative. Tell us what it is, Mags.
\end{dialogue}

I booted up stellarium and eyed Terry's monitor to read the time of capture. I entered it, panned straight up and scrolled way in.

\begin{dialogue}
  Maggie\: What's our per-pixel angular resolution again? [FILL ME IN], right?

  Terry\: You got it.

  Maggie\: Alright, looks like that pixel is very likely HD 4628.

  Frieda\: Ugh, one of those with the impersonal names.

  Anne\: Why don't we name it?
\end{dialogue}

We ended up naming it XAO's star, in honour of the observatory. Then an alarm started going off on Terry's phone.

\begin{dialogue}
  Terry\: Oh shit, Orbicomm is coming by in 10 minutes.

  Maggie\: We tested the timing system, right?

  Terry\: Fuck. We can do one now for like, five minutes, then we'd immediately have to set one for the orbicomm capture.
\end{dialogue}

I sighed and leaned back on the bench.

\begin{dialogue}
  Maggie\: Hm, well if the timing system doesn't work that gives us 5 minutes to fix it, which will consist of us running around like headless chickens, I think. How about we use the orbicomm capture to test the timing.

  Terry\: That sounds very mavericky, even for you.

  Maggie\: I'm just being realistic. Also I am not that mavericky.

  Anne\: You kinda are.

  Frieda\: Mhm

  Maggie\: Oh now you all gang up on me? Terry doesn't do error handling!
\end{dialogue}

The pair gasped and looked at Terry in mock-surprise. Terry held his chest.

\begin{dialogue}
  Terry\: Et Tu, Maggie?

  Maggie\: Feel the burn, buddy. Uh, we should really set that timer.

  Terry\: I was doing it while you were trying to think of a comeback.

  Maggie\: Yeesh, banter. 
\end{dialogue}

We were silent for a moment.

\begin{dialogue}
  Maggie\: Uh, you /did/ set it, right?

  Terry\: Yeah, yeah it's set. Counting down in the corner there.
\end{dialogue}

Orbcomm is a (now defunct) satellite communication network with [FILLMEIN] satellites operating in low earth orbit. One of them, Orbcomm-QL2, is by chance flying directly overhead tonight at [FILLMEIN]. Well, we had to move the imaging array about 40 meters south east from where we built it, and we'll need to move it back to image the anomaly, but other than that it's quite the stroke of luck.

The satellite is about 1.5 meters tall, and since low earth orbit is negligible compared to the radius of the earth this corresponds to a shadow that's 1.4 meters tall, a perfect size to test the array on.

It would have been neat to go outside and watch the shadow zoom over the imager, however we didn't do it for two reasons:

\begin{enumerate}
  \item the imager is an eight minute walk from the control room

  \item the shadow is only about $1\mathrm{m}^2$, basically imperceptible unless you're looking straight up at the light, and even then all you'd see is a brief dimming.
\end{enumerate}

So we kept inside the utility closet--I mean control room--and chatted while we watched the countdown. 

\begin{dialogue}
  Maggie\: Who wants to bet we'll see the outline of an unknown lost cosmonaut?

  Terry\: 30 bucks says he's waving.

  Anne\: Hmmm, I like those odds.

  Frieda\: 10 seconds.
\end{dialogue}

And then we were silent. Three, two, one, zero.

There was nothing for maybe a second or a half, then the images started appearing in their place. As it happened the first cameras to respond were /not/ the ones that the shadow fell on, our post-processing code is fast on blank images but slow on actual shadows. 

\interlude

Nearly every known astronomical body can be imaged by traditional astronomy, almost by definition. Of these bodies very few have /shadows/ that can even be imaged to begin with. And even then shadows don't have very many positives: they're blurry, noisy, and whereas light reflected or emitted from a body can be viewed from anywhere on the planet (save some exceptions, like southern or northern constellations) to find a shadow you have to either have a lot of luck or have a lot of money to travel. Because of these reasons shadow astronomy hasn't really taken off.

The anomaly (also known as Kordel's Object or The Kordel Object) is special in that it can be imaged through it's shadow, but the ``source'' of the shadow is yet to be found. The anomaly follows an extremely irregular and aperiodic orbit. The orbit, when projected onto the earth, overlaps itself so many times it resembles a pincushion. This orbit, a Lissajous curve, is earth synchronous, and based on speed an trajectory must be orbiting Lagrangian point $L_2$.

A Lagrangian point is essentially a saddle point between gravity wells. For the earth and the sun, it is about where the gravitational force of the sun dominates the gravitational force of the earth [VERIFY LATER]. There are five such points in the earth-sun system, two in front and behind the earth with respect to the sun ($L_1$ and $L_2$), one on the exact opposite side of the sun at earth distance ($L_3$), and two located to the left and right of the earth at a third of an orbital distance apart ($L_4$ and $L_5$) [VERIFY LATER].

The anomaly orbits $L_2$, about 4 times the distance from the earth to the moon, and has an orbital radius the same as the moon's to the earth. From this information we can calculate it's major length from it's shadow dimensions. The shadow is on average about 8m tall, which implies a length of ~2km. As mentioned before, it's orbit is highly unstable, unstable enough to collapse after at most a year. The anomaly has been around since it was discovered 6 years ago, so it is highly unlikely that it is a transient asteroid.

Even though we know it's position at any given moment, the anomaly refuses to be captured by ground or orbital based teloscopy. When we point a telescope at it, there's nothing. When we shoot Doppler radar at it, still nothing. The anomaly either absorbs fully or is totally transparent to visible light, infrared light, x-rays, radio waves, microwaves, and any other electromagnetic radiation I may have missed. Since it casts a shadow it /must/ be absorbent, not transparent. But, when predictions say it must be transiting a star or other body, no such transiting is visible. Either it is absorbent to the Light and transparent to everything else, or something very strange is going on.

To sum up. There are two things terribly wrong with the anomaly. First: it's orbit is extremely unstable, yet it has sustained it for years, and two: it's absorption spectra is inconsistent across observations.


\end{document}
